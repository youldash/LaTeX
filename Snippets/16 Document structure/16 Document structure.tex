%
%  Document Structures
%
%  Created by Mustafa Youldash on 2014-04-25.
%  Copyright (c) 2014 Mustafa YOULDASH. All rights reserved.
%
\documentclass[12pt,oneside,onecolumn,a4paper,final]{article}

% .......................................... Preamble

% Use utf-8 encoding for foreign characters
\usepackage[utf8]{inputenc}

% Colors
\usepackage{xcolor}

% .......................................... End

\begin{document}

	% .......................................... Top Matter
	\title{Document Structures}
	\author{Mustafa M. Youldash \\
		Department of Computer and Information Systems \\
		Umm Al-Qura University \\
		\texttt{mmyouldash@uqu.edu.sa} \\
		(+61) 432-109-876 $\;$ (+61) 3 9876-5432}
	\date{\today}
	\maketitle
	% .......................................... End

	% .......................................... Abstract
	\begin{abstract}
		The main point of writing a text is to \textcolor{red}{convey ideas}, information, or knowledge to the reader. The reader will understand the text better if these ideas are well-structured, and will see and feel this structure much better if the typographical form reflects the logical and semantic structure of the content.
	\end{abstract}
	% .......................................... End

	\section{The document environment}
	After the Document Class Declaration, the text of your document is enclosed between two commands which identify the beginning and end of the actual document \ldots

	\section{Preamble}
	The \texttt{preamble} is everything from the start of the Latex source file until the \\begin{document} command. It normally contains commands that affect the entire document.

	\section{Conclusion}
	\LaTeX\ is different from other typesetting systems in that you just have to tell it the logical and semantical structure of a text. It then derives the typographical form of the text according to the ``rules'' given in the document class file and in various style files. \LaTeX\ allows users to structure their documents with a variety of hierarchal constructs, including chapters, sections, subsections and paragraphs.

	\appendix
	\section{First Appendix}

\end{document}
