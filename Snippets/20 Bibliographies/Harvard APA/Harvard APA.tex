%
%  Bibliography using Harvard APA
%
%  Created by Mustafa Youldash on 2014-04-25.
%  Copyright (c) 2014 Mustafa YOULDASH. All rights reserved.
%
\documentclass[12pt,a4paper,oneside]{report}
% Alternative Options:
%	Paper Size: a4paper / a5paper / b5paper / letterpaper / legalpaper / executivepaper
% Duplex: oneside / twoside
% Base Font Size: 10pt / 11pt / 12pt

% .......................................... Preamble

%% Package for hyper referencing
% \usepackage{hyperref}

%% Packages for graphics and figures
\usepackage{graphicx} %%For loading graphic files
%\usepackage{subfig} %%Subfigures inside a figure
%\usepackage{tikz} %%Generate vector graphics from within LaTeX

%% The bibliography
%% ==> You need a file 'literature.bib' for this.
%% ==> You need to run BibTeX for this (Project | Properties... | Uses BibTeX)
% Citation and References macros for biblatex [APA biblatex style]
\usepackage[american]{babel}
% Required for language-specific quotation marks.
\usepackage{csquotes}
\usepackage{shortvrb}
\usepackage{ifthen}
% Using the 'authoryear' style in this example. Default is the 'numerical' style.
\usepackage[backend=biber,style=apa,apabackref=true]{biblatex}
\DeclareLanguageMapping{american}{american-apa}
\usepackage{xpatch}
\xpatchnameformat{labelname}{\ifciteseen}{\ifnumcomp{\value{listtotal}}{>}{\value{maxnames}}}{}{}
% \usepackage[style=authoryear]{biblatex}
% Using the database biblatex-examples.bib.
\addbibresource{Literature}
\usepackage{nameref}

%% Please note:
%% Images can be included using \includegraphics{filename}
%% resp. using the dialog in the Insert menu.
%% 
%% The mode "LaTeX => PDF" allows the following formats:
%%   .jpg  .png  .pdf  .mps
%% 
%% The modes "LaTeX => DVI", "LaTeX => PS" und "LaTeX => PS => PDF"
%% allow the following formats:
%%   .eps  .ps  .bmp  .pict  .pntg

%% Math packages
% \usepackage{amsmath}
% \usepackage{amsthm}
% \usepackage{amsfonts}

%% Line spacing
%\usepackage{setspace}
%\singlespacing        %% 1-spacing (default)
%\onehalfspacing       %% 1,5-spacing
%\doublespacing        %% 2-spacing

% .......................................... End

\begin{document}

	\pagestyle{empty} % No headings for the first pages.

	%% Title Page %%%%%%%%%%%%%%%%%%%%%%%%%%%%%%%%%%%%%%%%%%%%%%%
	%% ==> Write your text here or include other files.

	%% The simple version:
	\title{Bibliographies with \LaTeX}
	\author{M. Youldash}
	\date{\today}
	\maketitle

	\tableofcontents %Table of contents
	\cleardoublepage %The first chapter should start on an odd page.

	\pagestyle{plain}

	%% Chapters %%%%%%%%%%%%%%%%%%%%%%%%%%%%%%%%%%%%%%%%%%%%%%%%%
	\chapter{Introduction}
	\label{cha:introduction}



	\section{Harvard}
	\label{sec:harvard}
	Investigate the Harvard referencing package
	(\verb#\usepackage{harvard}#).



	\section{References}
	\label{sec:references}

	Using the commands \verb#\label{name}# and \verb#\ref{name}# you are able
	to use references in your document. Advantage: You do not need to think
	about numerations, because \LaTeX\ is doing that for you.

	For example, in section~\ref{sec:dividing} on
	age~\pageref{sec:dividing} hints for
	dividing large documents are given.

	Certainly, references do also work for tables, figures, formulas\ldots

	Please notice, that \LaTeX\ usually needs more than one run (mostly 2) to
	resolve those references correctly.

	Instead of WYSIWYG editors, typesetting systems like TeX or
	%LaTeX \cite{lamport94} can be used.

	% Or to be more specific: \cite[p. 215]{lamport94}.

	% \cite{lamport95} showed in 1995 something\ldots  \nocite{lamport95}.



	\section{BibTeX}
	\label{sec:BibTeX}

	\textcite{AbedonHymanThomas2003}

	\citeauthor{goossens93} noted an important thing about \ldots, \parencite{greenwade93}.



	\section{Dividing Large Documents}
	\label{sec:dividing}

	You can divide your \LaTeX-Document into an arbitrary number of \TeX-Files
	to avoid too big and therefore unhandy files (e.g. one file for every chapter).



	\section{Other}
	\label{sec:Other}

	\emph{http://scholar.google.com}

	%% <== End of Introduction
	%%%%%%%%%%%%%%%%%%%%%%%%%%%%%%%%%%%%%%%%%%%%%%%%%%%%%%%%%%%%%

	%%%%%%%%%%%%%%%%%%%%%%%%%%%%%%%%%%%%%%%%%%%%%%%%%%%%%%%%%%%%%
	%% BIBLIOGRAPHY AND OTHER LISTS
	%%%%%%%%%%%%%%%%%%%%%%%%%%%%%%%%%%%%%%%%%%%%%%%%%%%%%%%%%%%%%
	%% A small distance to the other stuff in the table of contents (toc)
	\addtocontents{toc}{\protect\vspace*{\baselineskip}}

	%%
	%% Embed system
	%%
	%\begin{thebibliography}{9}
	% 
	%\bibitem{myCite2009}
	%	Mohammed Saeed,
	%	\textit{A good book on LaTeX},
	%	RMIT University Piblishers,
	%	1st Edition,
	%	2009.
	%	
	%\bibitem{lamport94}
	%  Leslie Lamport,
	%  \emph{\LaTeX: A Document Preparation System}.
	%  Addison Wesley, Massachusetts,
	%  2nd Edition,
	%  1994.
	% 
	%\end{thebibliography}


	%% The List of Figures
	\clearpage
	\addcontentsline{toc}{chapter}{List of Figures}
	\listoffigures

	%% The List of Tables
	\clearpage
	\addcontentsline{toc}{chapter}{List of Tables}
	\listoftables

	%%%%%%%%%%%%%%%%%%%%%%%%%%%%%%%%%%%%%%%%%%%%%%%%%%%%%%%%%%%%%
	%% APPENDICES
	%%%%%%%%%%%%%%%%%%%%%%%%%%%%%%%%%%%%%%%%%%%%%%%%%%%%%%%%%%%%%
	\appendix

	% The list of references is printed by way of \printbibliography.
	\printbibliography[title=References]

\end{document}
